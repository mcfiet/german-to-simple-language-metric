\subsection{Code und Daten}
Der Code der in diesem Projekt verwendet wurde ist unter \href{https://github.com/mcfiet/genai-project}{https://github.com/mcfiet/genai-project} zu finden.

\subsection{Motivation}
\begin{tcolorbox}[boxrule=0pt,sharp corners,enhanced,borderline west={2pt}{0pt}{blue}]
    \textbf{Für die meisten ein ganz normaler Satz – für andere das Ende der Teilhabe.}
\end{tcolorbox}

Beim Thema Barrierefreiheit denken viele an Rampen, Aufzüge und automatische Türen. Doch durch die Digitalisierung und die zunehmende Verlagerung von Informationen ins Internet reicht das längst nicht mehr aus.

Auch Sprache selbst kann eine Barriere darstellen. Menschen werden durch komplexe Texte, Fachbegriffe und lange Satzkonstruktionen vom Zugang zu Informationen ausgeschlossen. Dies betrifft nicht nur Personen mit geistigen Einschränkungen, sondern auch ältere Menschen, Menschen mit geringen Deutschkenntnissen oder Situationen, in denen schnelle und klare Information entscheidend ist. Leichte Sprache nützt daher weit mehr Menschen, als die die darauf angewiesen sind.

\subsection{Problemstellung}
Leichte Sprache ist entscheidend für barrierefreie Kommunikation, ihre Erstellung ist jedoch aufwendig und überwiegend manuell. Für das Deutsche fehlen zugleich robuste, datengetriebene Verfahren, die zuverlässig zwischen Leichter Sprache und Standardsprache unterscheiden oder die Qualität automatisierter Vereinfachung bewerten. Die Anforderungen sind hoch: Bereits einzelne schwer verständliche Wörter können den Leseprozess abbrechen.

\subsection{Ziel dieser Arbeit}
Ziel dieser Arbeit ist es, eine zuverlässige Metrik zu entwickeln, mit der bewertet werden kann, ob ein Satz oder eine Textsequenz Leichte Sprache ist oder nicht. Diese Metrik ist insbesondere für das spätere Training automatischer Übersetzer wichtig, da sie die Qualität der Vereinfachung objektiv messbar macht. Dazu wird ein reproduzierbarer Datensatz aufgebaut, eine Modellpipeline zur Klassifikation entwickelt und die Metrik auf synthetischen und realitätsnahen Daten getestet.
