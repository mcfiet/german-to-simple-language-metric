\subsection{Code und Daten}
Der Code der in diesem Projekt verwendet wurde ist unter \href{https://github.com/mcfiet/genai-project}{https://github.com/mcfiet/genai-project} zu finden.

\subsection{Motivation}
\begin{tcolorbox}[boxrule=0pt,sharp corners,enhanced,borderline west={2pt}{0pt}{blue}]
    \textbf{Für die meisten ein ganz normaler Satz – für andere das Ende der Teilhabe.}
\end{tcolorbox}

Beim Thema Barrierefreiheit denken viele an Rampen, Aufzüge und automatische Türen. Doch durch die Digitalisierung und die zunehmende Verlagerung von Informationen ins Internet reicht das längst nicht mehr aus.

Auch Sprache selbst kann eine Barriere darstellen. Menschen werden durch komplexe Texte, Fachbegriffe und lange Satzkonstruktionen vom Zugang zu Informationen ausgeschlossen. Dies betrifft nicht nur Personen mit geistigen Einschränkungen, sondern auch ältere Menschen, Menschen mit geringen Deutschkenntnissen oder Situationen, in denen schnelle und klare Information entscheidend ist. Leichte Sprache nützt daher weit mehr Menschen, als die die darauf angewiesen sind.

\subsection{Problemstellung}
Leichte Sprache ist ein zentraler Bestandteil barrierefreier Kommunikation. Sie ermöglicht Menschen mit kognitiven Einschränkungen, älteren Personen oder Menschen mit geringen Deutschkenntnissen den Zugang zu Informationen, die ihnen sonst verschlossen blieben. Während Behörden und Institutionen zunehmend Texte in Leichter Sprache anbieten, ist die Erstellung weiterhin stark von manueller Übersetzung abhängig. Automatisierte Verfahren könnten diesen Prozess unterstützen, sind im Deutschen jedoch bislang kaum etabliert.

Im Rahmen dieser Arbeit fanden Gespräche mit der Lebenshilfe Kiel statt. Dabei wurde deutlich, wie hoch die Anforderungen an Leichte Sprache tatsächlich sind: Schon ein einziges schwer verständliches Wort kann dazu führen, dass betroffene Menschen den gesamten Leseprozess abbrechen. Anders als geübte Leserinnen und Leser verfügen sie nicht über Strategien, um Barrieren zu umgehen oder über komplizierte Passagen hinwegzulesen.

Die Lebenshilfe selbst übersetzt regelmäßig Texte für unterschiedliche Auftraggeber in Leichte Sprache. Diese praktische Erfahrung verdeutlicht, wie groß der Bedarf ist – und war zugleich ein wesentlicher Ausgangspunkt für diese Arbeit. Ziel ist es, ein Verfahren zu entwickeln, das Texte automatisch in Leichte Sprache überträgt und damit einen Beitrag zur Barrierefreiheit leistet.

\subsection{Ziel dieser Arbeit}


