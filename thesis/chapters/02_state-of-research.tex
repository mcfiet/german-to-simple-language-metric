Forschung zur automatischen Bewertung oder Metriken für Leichte Sprache ist bislang kaum vorhanden. Das macht das Thema zugleich spannender und ist ein zentraler Antrieb dieser Arbeit: Es gibt derzeit keine etablierte, datengetriebene Metrik für Leichte Sprache im Deutschen, die als Training- oder Bewertungssignal genutzt wird. Der größte Teil der Literatur konzentriert sich stattdessen auf die Erzeugung und Ausrichtung von Korpora sowie auf Benchmarking in der Textvereinfachung. Einen wichtigen Beitrag für deutsche Ressourcen liefern \textcite{toborek-etal-2023-new} mit einem alignierten Korpus für Einfaches Deutsch. Auf breiterer Ebene liefert \textcite{kloeser2024} eine aktuelle Übersicht zu multilingualer Textvereinfachung, Benchmarks und Methoden. Abseits dieser Arbeiten zur Datengrundlage gibt es bisher kaum Forschung, die explizit Metriken oder Bewertungssignale für Leichte Sprache entwickelt und evaluiert. Genau hier setzt diese Arbeit an.
